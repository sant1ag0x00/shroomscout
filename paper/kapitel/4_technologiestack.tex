\documentclass[../main.tex]{subfiles} % ~1800 Worte

\begin{document}

\subsection{Uebergreifende Technologien} % Lauritz

\subsubsection{VSCode}

Visual Studio Code, kurz VSCode, ist eine integrierte Entwicklungsumgebung (IDE), die von Microsoft entwickelt wurde.
Eine IDE ist eine Software, die umfassende Funktionen zur Softwareentwicklung unter einer Oberfläche bündelt, einschließlich
eines Quelltext-Editors, Debugging-Tools und oft auch Möglichkeiten zur Versionskontrolle. Die Entscheidung für VSCode als
primäre Entwicklungsumgebung für `ShroomScout' fiel aufgrund mehrerer Faktoren. Zum einen bringen beide Entwickler bereits
umfassende Erfahrung mit VSCode aus früheren Projekten mit, was eine effiziente Arbeit ermöglicht. Zum anderen bietet VSCode
eine einfache Erweiterbarkeit durch zahlreiche Extensions, die speziell für die Arbeit mit Angular, Git und anderen relevanten
Technologien konzipiert sind. Diese Erweiterungen erleichtern die Entwicklung erheblich, indem sie zusätzliche Funktionen wie
Syntaxhervorhebung, Codevervollständigung und direkte Integration mit Git bereitstellen.

\subsubsection{Git \& GitHub}

Git ist ein weit verbreitetes System zur Versionskontrolle, das es Entwicklern ermöglicht, Änderungen am Quellcode nachzuverfolgen
und die Zusammenarbeit in Entwicklungsteams zu erleichtern. Git unterstützt die verteilte Arbeit, indem es jedem Entwickler ermöglicht,
eine vollständige Kopie des Repositorys lokal zu führen, inklusive der gesamten Historie der Codeänderungen. Für `ShroomScout' wurde
Git genutzt, um den Entwicklungsprozess zu verwalten und Änderungen effizient zu koordinieren. Das Projekt wurde auf GitHub gehostet,
einer Plattform, die kostenlose Hosting-Services für Git-Repositorys bietet. GitHub erleichtert die Zusammenarbeit, indem es Werkzeuge
für die Problemverfolgung, Code-Reviews und die Verwaltung von Pull Requests bereitstellt. Die Nutzung von GitHub als zentrale Plattform
trug zur reibungslosen Koordination zwischen den Entwicklern bei und ermöglichte eine transparente und nachvollziehbare Projektentwicklung.

\subsubsection{Open Street Map}

Open Street Map (OSM) ist ein kollaboratives Projekt zur Erstellung einer frei bearbeitbaren Weltkarte. Im Gegensatz zu kommerziellen
Kartenanbietern wie Google Maps bietet OSM den Vorteil, dass die Daten unter einer offenen Lizenz stehen, was die freie Nutzung und
Anpassung der Karteninformationen ermöglicht. Für `ShroomScout' wurde OSM als Grundlage für die interaktive Karte gewählt, auf der Nutzer
Pilzfundorte eintragen können. Ein entscheidender Vorteil von OSM ist die Möglichkeit, eigene Layer zu erstellen und zu verwalten, was es
ermöglicht, Pilzfundorte präzise zu dokumentieren und visuell hervorzuheben. Diese Flexibilität und die offene Datenlizenz machen OSM zu
einer idealen Wahl für Projekte, die auf detaillierte geographische Informationen angewiesen sind und gleichzeitig die Kontrolle über die
Darstellung und Verwaltung der Kartendaten behalten möchten.

\subsection{Datenbankloesung} % Tim

\subsubsection{SQLite}

\subsubsection{Tabellenstruktur}

\subsection{Backend Technologien} % Tim

\subsubsection{FastAPI mit Python}

\subsubsection{Uvicorn ASGI Server}

\subsection{Frontend Technologien} % Lauritz

\subsubsection{Angular & Typescript}

Angular ist ein modernes, leistungsfähiges Frontend-Framework, das von Google entwickelt wurde und für die Erstellung dynamischer Single-Page 
Applications (SPAs) eingesetzt wird. SPAs laden die notwendigen HTML-, JavaScript- und CSS-Codes bei der ersten Anfrage an den Server und 
aktualisieren anschließend den Inhalt dynamisch ohne Seitenneuladung. Dieser Ansatz ermöglicht eine nahtlose Benutzererfahrung, ähnlich wie bei 
einer Desktop-Anwendung, und bietet Vorteile wie verbesserte Performance, schnelle Reaktionszeiten und reduzierte Serveranfragen.

Des weiteren basiert Angular auf TypeScript, einer typisierten Superset-Sprache von JavaScript, die in JavaScript umgewandelt wird. Sie führt 
objektorientierte Programmierkonzepte, wie optionale statische Typisierung ein, was zu einer besseren Codequalität, einfacherer Wartbarkeit und 
erhöhter Entwicklerproduktivität führt. TypeScript erweitert JavaScript außerdem um zusätzliche Syntax für die Definition von Typen und unterstützt 
somit die Entwicklung großer Anwendungen, indem es frühzeitige Fehlererkennung, bessere Refactoring-Fähigkeiten und eine autokomplette Funktion bietet.

\subsubsection{Komponenten \& Services}

Die Architektur von Angular ist komponentenbasiert, was bedeutet, dass Anwendungen aus Komponenten aufgebaut sind, die
jeweils einen Teil der Benutzeroberfläche darstellen. Diese Komponenten können unabhängig voneinander entwickelt, getestet und in der
Anwendung wiederverwendet werden, was die Entwicklungseffizienz und -geschwindigkeit erhöht. Eine Komponente in Angular besteht aus einer 
Klasse, die mit einem @Component-Dekorator versehen ist. Dieser Dekorator definiert zugehörige Metadaten wie das Template (HTML), Styles (CSS) 
und weitere Selektoren, wodurch eine Komponente die Kontrolle über einen bestimmten Bereich des Bildschirms in der Anwendung erhält.

Im Gegensatz zu Komponenten fokussieren sich Services in Angular auf die Implementierung von Geschäftslogik und die Bereitstellung von Daten. 
Sie sind unabhängige Klassen, markiert mit einem @Injectable-Dekorator, und dienen dazu, wiederverwendbare Logik oder Datenzugriffsmethoden zu 
kapseln. Services können in Komponenten oder anderen Services injiziert werden, wodurch eine lose Kopplung und eine höhere Wiederverwendbarkeit 
der Codebestandteile gefördert wird.

Folgende Bestandteile der GUI wurden in Komponenten gekapselt:

\begin{itemize}

	\item \textbf{Navigationsleiste:}
	      Die Komponente `nav-bar' umfasst die Navigationsleiste am oberen Bildschirmrand. Sie ist immer zu sehen und bietet dem User die Möglichkeit,
        über eine Schaltfläche zur Eingabemaske oder über das Logo zurück zur Startseite zu navigieren. Dazu wird lediglich zur URL `.../register' für die 
        Eingabemaske und zurück zur Root-URL für den Live-Feed navigiert. Im Routing Modul (`app-routing.module.ts') sind für diese Routen dann die
        zugehörigen Komponenten deklariert, welche in der App Komponente (`app.component.ts') an Stelle des router-outlet-Tags angezeigt werden.
        Dabei wird auch klar, dass die Karte immer zu sehen ist und lediglich der Teil daneben durch die entsprechende Komponente je nach der Route 
        befüllt wird. 

  \item \textbf{Karte:}
        In der Komponente `map' wird die Karte wird mithilfe des `leaflet'-Pakets (siehe `package.json') mit mit sogenannten `tiles', also kleinen 
        Teilen der Weltkarte, befüllt. Diese werden von der OpenStreetMap Tile-Provider URL geladen. Außerdem sorgt die Komponente dafür, dass die Karte 
        anklickbar wird, sollte die URL auf `/register' enden, sprich wenn die Eingabemaske angezeigt wird. Klick-Events durch den User werden vom Marker 
        Service verwaltet.

  \item \textbf{Live-Feed:}
        Der Live-Feed besteht aus einer Angular Material Card (`<mat-card>'). Angular Material ist eine Bibliothek an vorgefertigten GUI-Komponenten,
        welche bereits ansprechend aussehen. Diese wurden überall, wo es möglich war, verwendet, um sich nicht mehr um das CSS-Styling kümmern zu müssen.
        In der Komponente werden Nachrichten dargestellt, welche sie vom Message Service erhält. Das tut sie, indem sie mithilfe der async-Pipe (`| async') 
        das Nachrichten-Observable abonniert und somit über Änderungen automatisch informiert wird. 

  \item \textbf{Eingabemaske:}
        Die Eingabemaske besteht ebenfalls aus Angular Material Komponenten für die entsprechenden Eingabefelder und das Drop-Down-Menü. Außerdem wird
        hier beim Eintragen eines Pilzes zum Einen der Marker Service zum persistieren des Markers und der Message Service zum Absenden einer neuen 
        Nachricht an den Live-Feed genutzt.

\end{itemize}

Zwischen diesen Komponenten müssen an mehreren Stellen Daten ausgetauscht werden. Dazu dienen folgende Services:

\begin{itemize}

	\item \textbf{Message Service:}
	      Der Message Service ist für die im Live-Feed angezeigten Nachrichten zuständig. Somit enthält er eine Liste aller Nachrichten, sowie eine Methode,
        welche zum Hinzufügen einer neuen Nachricht dient. Die aktuellen Nachrichten werden dann über einen sogennanten `Observable Stream', also einem
        sich automatisch aktualisierenden Datenfluss bereitgestellt. Die Eingabemaske sendet eine neue Nachricht an das Observable und der Live-Feed
        abonniert dieses Observable, wodurch es die neue Nachrichtenliste automatisch erhält, sobald diese sich ändert. Dadurch wird eine effektive 
        Brücke zwischen den beiden Komponenten geschlagen.

  \item \textbf{Marker Service:}
        Der Marker Service dient der Verwaltung der Pilzfundort-Markierungen auf der Karte. Er erhält die angezeigte Karteninstanz und kann dadurch auf 
        dieser temporäre Markierungen setzen. Außerdem enthält er Methoden zum Persistieren eines Markers, wenn der User einen Pilz einträgt. Somit ist 
        auch dieser Service essentiell, um das Zusammenspiel zwischen der Karten- und Eingabemaskenkomponente zu ermöglichen.

\end{itemize}

\subsubsection{Backend Kommunikation}  % Tim

\subsubsection{Responsive Design}

Die Anpassung der HTML Templates an kleinere Bildschirmgrößen wurde ausschließlich in den CSS-Dateien über die `@media'-Query geregelt. Dort können für
selbst festgelegte Bildschirmbreiten und -höhen andere CSS-Styles festgelegt werden, als normalerweise gelten. Somit wurden durch wiederholtes Anpassen
und Ausprobieren alle Teile der Applikation so gestaltet, dass sie bei jeder Bildschirmgröße sichtbar und sinnvoll platziert sind.

\end{document}
