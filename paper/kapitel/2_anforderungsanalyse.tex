\documentclass[../main.tex]{subfiles}
%~1400 Worte
\begin{document}

\subsection{Benutzeranforderungen als Use Cases} %Lauritz
Bei der Entwicklung von Softwareprojekten, wie der Webanwendung "ShroomScout", ist die sorgfältige Definition von 
Benutzeranforderungen ein entscheidender Schritt. Eine bewährte Strategie zur Strukturierung und Priorisierung dieser 
Anforderungen ist die Erstellung von Use Cases. Use Cases beschreiben typische Interaktionen zwischen dem Nutzer und 
dem System und dienen dazu, klare und nachvollziehbare Anforderungen zu formulieren. Im Folgenden werden die zentralen 
Use Cases für "ShroomScout" detailliert beschrieben, um die grundlegenden Benutzeranforderungen zu veranschaulichen.

1. Karteninteraktion
Die Karte ist das zentrale Element von "ShroomScout" und ermöglicht es den Nutzern, geographisch relevante Informationen 
visuell zu erfassen. Nutzer können auf der Karte navigieren, indem sie hin- und herscrollen sowie hinein- und herauszoomen, 
um unterschiedliche Bereiche und Detailstufen zu erkunden. Diese Funktionen sind essenziell, um die Suche nach Pilzstandorten 
intuitiv und effizient zu gestalten. Die Interaktion mit der Karte ist so konzipiert, dass Nutzer leicht den gewünschten 
Ausschnitt finden können, was die Benutzerfreundlichkeit der Anwendung maßgeblich erhöht.

2. Pilzfund Dokumentation
Ein weiterer zentraler Use Case ist das Eintragen eines Pilzfunds durch den Nutzer. Dieser Prozess beginnt mit dem Klick 
auf die Schaltfläche "Pilz eintragen" und umfasst mehrere Schritte:

Eingabe von Pilzname und Umgebung: 
Der Nutzer gibt den Namen des Pilzes ein, unterstützt durch eine automatische Vervollständigung, um Tippfehler zu vermeiden 
und den Prozess zu beschleunigen. Zusätzlich wird die Umgebung des Pilzfunds (z.B. Wiese, Eiche) erfasst.

Anzeigen eines Bildes des Pilzes: Nach der Eingabe des Pilznamens wird automatisch ein Bild des entsprechenden Pilzes angezeigt. 
Dies dient der visuellen Bestätigung und hilft, Verwechslungen zu vermeiden.

Markieren des Fundorts auf der Karte: Der Nutzer platziert einen Marker auf der Karte, um den genauen Fundort des Pilzes zu 
dokumentieren.

Hover-Effekt über Marker: Nachdem der Pilz eingetragen wurde, kann der Nutzer über den Marker auf der Karte hovern, um den 
Namen des Pilzes einzusehen. Dies ermöglicht eine schnelle Identifikation der eingetragenen Funde direkt auf der Karte.

3. Anzeige eines Live-Feeds
Der Live Feed stellt eine Auflistung der jüngsten Pilzfunde dar und wird kontinuierlich aktualisiert. Obwohl Nutzer mit dem 
Live Feed nicht direkt interagieren können, dient er als wichtige Informationsquelle und Inspirationsquelle für die Community. 
Die Anzeige der neuesten Funde fördert das Gemeinschaftsgefühl und motiviert Nutzer, eigene Entdeckungen zu teilen. Obwohl der 
Live Feed keine direkte Nutzerinteraktion beinhaltet, ist er ein wesentlicher Bestandteil des Nutzererlebnisses und trägt zur 
Dynamik der Anwendung bei.

Zusammenfassend bilden diese Use Cases das Fundament für die Interaktion der Nutzer mit "ShroomScout" und definieren klare A
nforderungen an die Funktionalität und Benutzerfreundlichkeit der Anwendung. Durch die detaillierte Ausarbeitung dieser Use 
Cases wird sichergestellt, dass die Entwicklung von "ShroomScout" den Bedürfnissen und Erwartungen der Zielgruppe entspricht 
und eine intuitive, effiziente und ansprechende Benutzererfahrung bietet.

\subsection{Funktionale Anforderungen} %Tim

\subsection{Nicht-funktionale Anforderungen} %Lauritz
Neben den funktionalen Anforderungen, die spezifische Aktionen und Verhaltensweisen des Systems beschreiben, spielen 
nicht-funktionale Anforderungen eine entscheidende Rolle für den Erfolg eines Softwareprojekts. Sie definieren die 
Qualitätsattribute der Anwendung und beeinflussen maßgeblich die Nutzerzufriedenheit. Für die Webanwendung "ShroomScout" 
lassen sich folgende nicht-funktionale Anforderungen festhalten:

Einfachheit und Benutzerfreundlichkeit
"ShroomScout" soll durch eine intuitive Bedienbarkeit bestechen. Die Anwendung muss so gestaltet sein, dass Nutzer mit 
minimalen Aufwand und ohne Vorkenntnisse die Kernfunktionalitäten nutzen können. Dazu gehört eine klare und verständliche 
Nutzeroberfläche, die es dem Anwender ermöglicht, ohne umständliche Navigation oder komplizierte Prozesse Pilze zu finden 
und einzutragen.

Responsive Design
Angesichts der Tatsache, dass Nutzer "ShroomScout" häufig im Freien und damit auf mobilen Geräten nutzen werden, ist ein 
responsive Design unerlässlich. Die Anwendung muss sich automatisch an verschiedene Bildschirmgrößen und -auflösungen 
anpassen, um auf Smartphones, Tablets und Desktop-Computern gleichermaßen gut bedienbar zu sein. Dies gewährleistet eine 
optimale Benutzererfahrung unabhängig vom Endgerät.

Schnelle Ladezeiten
Für eine positive Nutzererfahrung sind kurze Ladezeiten von großer Bedeutung. "ShroomScout" sollte so optimiert sein, 
dass die Anwendung auch bei langsamer Internetverbindung schnell lädt. Dies ist besonders wichtig, da Nutzer die Anwendung 
möglicherweise in Gebieten mit schlechter Netzabdeckung verwenden.

Sicherheit und Datenschutz
Die Sicherheit persönlicher Daten und die Wahrung der Privatsphäre der Nutzer sind essenzielle Anforderungen. "ShroomScout" 
muss sicherstellen, dass alle Nutzerdaten, insbesondere Standortinformationen und persönliche Informationen, gemäß den 
geltenden Datenschutzrichtlinien behandelt und geschützt werden.

Skalierbarkeit
Die Anwendung muss in der Lage sein, mit einer zunehmenden Anzahl von Nutzern und Datenmengen zu skalieren. Dies stellt 
sicher, dass "ShroomScout" auch bei wachsender Beliebtheit und steigenden Anforderungen stabil und performant bleibt.

Barrierefreiheit
"ShroomScout" sollte so gestaltet sein, dass die Anwendung auch für Menschen mit Behinderungen zugänglich ist. Dies 
umfasst beispielsweise die Implementierung von Screenreader-Unterstützung und die Anpassung von Farbkontrasten für eine 
bessere Lesbarkeit.

Mehrsprachigkeit
Um eine breite Nutzerbasis anzusprechen und die Anwendung auch für nicht deutschsprachige Pilzsammler attraktiv zu machen, 
sollte "ShroomScout" die Möglichkeit bieten, die Benutzeroberfläche in verschiedenen Sprachen darzustellen.

Durch die Berücksichtigung dieser nicht-funktionalen Anforderungen in der Entwicklungsphase wird sichergestellt, dass 
"ShroomScout" nicht nur funktionell den Bedürfnissen der Nutzer entspricht, sondern auch in Bezug auf Qualität, 
Sicherheit und Benutzerfreundlichkeit höchsten Standards genügt.

\end{document}
